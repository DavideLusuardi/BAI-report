\section{Conclusions}
Both are very good approaches and managed to find a way to play the game well. Even
though, across multiple runs, NEAT seems to obtain more stable results, GP has
demonstrated to have a bigger potential thanks to the tree-structured individuals and overall
manages to reach very good results.

One of the issues that we encountered using GP is how to manage the different input types:
one may implement functions taking into account inputs of different types or, alternatively,
one can use a strongly typed primitive set as we do.

Another issue we have is that the execution of the algorithms takes a lot of time and
computational resources, especially when the frame threshold is reached. Due to this, the
number of runs and individuals should be limited.

In the end, we learned a lot about this field from this project, in particular how it is possible to
apply bio-inspired techniques to real-world problems and how to benefit from it. They have
proved to be a valid alternative to classic back-propagation algorithms for ANN and in this
case we have finally learned new ways to do a kind of reinforcement learning using the
fitness as a reward.
