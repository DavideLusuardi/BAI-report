\section{Conclusions}
As demonstrated, NEAT and GP are very good approaches able to find a way to play the game well. Even
though, across multiple runs, NEAT seems to obtain more stable results, GP has
proved to have a bigger potential thanks to the tree-structured individuals and overall
manages to reach very good results.

One of the issues that we encountered using GP is how to manage the different input types:
one may implement functions taking into account inputs of different types or, alternatively,
can use a strongly typed primitive set as we have done.

Another issue is that the execution of the algorithms takes a lot of time and
computational resources, especially when the frame threshold is reached. Due to this, the
number of runs and individuals should be limited.

In the end, with this project we learned a lot about this field, in particular how it is possible to
apply bio-inspired techniques to real-world problems and benefit from them. They have
proved to be a valid alternative to classic back-propagation algorithms for ANN and we have learned how to
apply them to Reinforcement Learning tasks using the fitness as a reward.
