\section{Introduction}
\IEEEPARstart{C}{omputer games} have been linked with artificial intelligence (AI) from its inception.
Computer games implement rich and complex environments that usually require a real-time
response by the player and a dynamic strategy with an adaptive behavior in order to reach a
good score or win depending on the kind of game.


Artificial Neural Networks (ANNs) are computational learning systems inspired by biological
neural networks that use a network of functions to understand and translate some input data
into a desired output. An ANN is based on a collection of connected units or nodes called
artificial neurons, each neuron may be in principle connected to all the others. We can
differentiate the neurons in 3 categories: input neurons, which receives the input data from 
the environment; hidden neurons, which elaborates the input data; output neurons, which outputs the desired output.

They are being deployed in a variety of tasks including playing computer games, but a core
problem with the standard approach for training them is that we do not know a priori the best
structure and we may have to try a lot of different architectures before finding one that
provides good results. NeuroEvolution of Augmenting Topologies (NEAT) \cite{NEAT} is a genetic
algorithm which attempts to simultaneously learn weight values and an appropriate topology
for a neural network. In our implementation \cite{repository}, NEAT is used to evolve an ANN that is
evaluated based on the current inputs to produce the output that encodes what the agent has to do.

Genetic Programming (GP) \cite{GP} is a method used to generate computer programs. Starting
from an initial population of programs, the GP algorithm will evolve them in order to solve
predescribed automatic programming and machine learning problems. In our implementation
\cite{repository}, GP evolves computer programs represented as tree structures, that are evaluated
recursively based on the current inputs to produce an output that says what the agent has to do.
